%!TEX TS-program = xelatex
%!TEX encoding = UTF-8 Unicode

%-------------------------------------------------------------------------------
% CONFIGURATIONS
%-------------------------------------------------------------------------------
% A4 paper size by default, use 'letterpaper' for US letter
\documentclass[12pt, a4paper]{awesome-cv_turkish}

% Configure page margins with geometry
\geometry{left=2.4cm, top=1.8cm, right=2.4cm, bottom=1.8cm, footskip=.5cm}

% Specify the location of the included fonts
\fontdir[fonts/]

% Color for highlights
% Awesome Colors: awesome-emerald, awesome-skyblue, awesome-red, awesome-pink, awesome-orange
%                 awesome-nephritis, awesome-concrete, awesome-darknight
\colorlet{awesome}{awesome-skyblue}
% Uncomment if you would like to specify your own color
% \definecolor{awesome}{HTML}{003865}

% Colors for text
% Uncomment if you would like to specify your own color
% \definecolor{darktext}{HTML}{414141}
% \definecolor{text}{HTML}{333333}
% \definecolor{graytext}{HTML}{5D5D5D}
% \definecolor{lighttext}{HTML}{999999}

% Set false if you don't want to highlight section with awesome color
\setbool{acvSectionColorHighlight}{true}

% If you would like to change the social information separator from a pipe (|) to something else
\renewcommand{\acvHeaderSocialSep}{\quad\textbar\quad}
%-------------------------------------------------------------------------------
%	PERSONAL INFORMATION
%	Comment any of the lines below if they are not required
%-------------------------------------------------------------------------------
% Available options: circle|rectangle,edge/noedge,left/right
\photo[rectangle,edge,right]{./profile}
\name{Şemsi Yiğit}{Özgümüş}
\position{Bilgisayar Mühendisi }
%{\enskip\cdotp\enskip}
%\address{}
\mobile{(+90) 555-452-11-57}
\email{yigitozgumus1@gmail.com}
\homepage{www.yigitozgumus.com}
\github{yigitozgumus}
\linkedin{yigitozgumus}
% \gitlab{gitlab-id}
% \stackoverflow{SO-id}{SO-name}
% \twitter{@twit}
\skype{yigit.ozgumus}
% \reddit{reddit-id}
% \medium{madium-id}
% \googlescholar{googlescholar-id}{name-to-display}
%% \firstname and \lastname will be used
% \googlescholar{googlescholar-id}{}
% \extrainfo{extra informations}
%\quote{``Be the change that you want to see in the world."}

%-------------------------------------------------------------------------------
\begin{document}


% Print the header with above personal informations
% Give optional argument to change alignment(C: center, L: left, R: right)
\makecvheader[L]

% Print the footer with 3 arguments(<left>, <center>, <right>)
% Leave any of these blank if they are not needed
\makecvfooter
  {}%{\today}
  {Şemsİ Yİğİt Özgümüş~~~·~~~ Öz geçmİş}
  {\thepage}

%-------------------------------------------------------------------------------
%	CV/RESUME CONTENT
%	Each section is imported separately, open each file in turn to modify content
%-------------------------------------------------------------------------------
%-------------------------------------------------------------------------------
%	SECTION TITLE
%-------------------------------------------------------------------------------
\cvsection{Özet}
%-------------------------------------------------------------------------------
%	CONTENT
%-------------------------------------------------------------------------------

\begin{cvparagraph}
Milano Teknik Üniversitesi'nin Bilgisayar Bilimi ve Mühendisliği Bölümü'nün yüksek lisans
programından mezun oldum. Veri mühendisliği ve makine öğrenmesinin uygulama
entegrasyonu öncelikli olarak çalışmak istediğim dallardan. Ölçeklenebilir sistem mimarisi, 
fonksiyonel programlama ve bulut programlaması kendimi geliştirmek istediğim alanlardan bazıları. 
Teknik zorluklara daha optimize çözümler üretmeye ilgi duyarım. Problem çözümüne yönelik yeni
teknolojileri ve araçları öğrenmekten keyif alırım.
\end{cvparagraph}
%-------------------------------------------------------------------------------
%	SECTION TITLE
%-------------------------------------------------------------------------------
\cvsection{Eğitim}
%-------------------------------------------------------------------------------
%	CONTENT
%-------------------------------------------------------------------------------
\begin{cventries}
%---------------------------------------------------------
  \cventry
    {Yüksek Lisans, Bilgisayar Bilimi ve Mühendisliği} % Degree
    {Milano Teknik Üniversitesi (Politecnico di Milano )} % Institution
    {Mİlano, İtalya} % Location
    {Eylül 2017 - Temmuz 2019} % Date(s)
    {
      \begin{cvitems} % Description(s) bullet points
        \item{\textbf{Tez Konusu}: \href{https://github.com/yigitozgumus/Polimi_Thesis/blob/master/Documentation/Thesis/2019_7_Ozgumus_Semsi_Yigit.pdf}{Adversarially Learned Anomaly Detection Using Generative Adversarial Networks} | Danışman: Giacomo Boracchi}
        \item {\textbf{Alınan Dersler}: Derin Öğrenme, Veri ve Metin Madenciliği, Esnek Programlama, Doğal Dil İşleme, Öneri Sistemleri, Bilgisayar Görüşü, Paralel Programlama, Bilgisayar Güvenliği, Programlama Altyapısı, Model Belirlemesi ve Veri Analizi}
      \end{cvitems}
    }
%---------------------------------------------------------
%---------------------------------------------------------
\cventry
{Lisans, Bilgisayar Mühendisliği } % Degree
{Boğaziçi Üniversitesi (Bogazici University)} % Institution
{İstanbul, Türkİye} % Location
{Eylül 2012 - Haziran 2017} % Date(s)
{
  \begin{cvitems} % Description(s) bullet points
    \item{\textbf{Mezuniyet Projesi} : Parodi Resim Belirlemesi}
    \item {\textbf{Alınan Dersler}:	Robotiğe Giriş, Yapay Zekanın Prensipleri,
    Tutarlılık ve Dağıtımlı Programlama, Sistem Programlaması, Makine Öğrenmesi, Bayesian İstatistiği ve Makine Öğrenmesi}
  \end{cvitems}
}
%---------------------------------------------------------
\end{cventries}
%-------------------------------------------------------------------------------
%	SECTION TITLE
%-------------------------------------------------------------------------------
\cvsection{Çalışma Tecrübesi}
%-------------------------------------------------------------------------------
%	CONTENT
%-------------------------------------------------------------------------------
\begin{cventries}
%---------------------------------------------------------
\cventry{Android Yazılım Geliştirme Stajı}
{Monitise MEA (Commencis)}
{İstanbul, Türkİye}
{Temmuz 2016 - Ağustos 2016 }
{
  \begin{cvitems}
    \item {Android işletim sistemi öğrenilerek lokalizasyon, açık kaynak programlama arayüzü ve Google'ın hizmetlerini entegre kullanan bir uygulama geliştirildi.}
    \item {Gradle ve AndroidManifest dosyaları kullanılarak farklı aşamalardaki derlenmiş uygulamaları test eden bir Python programı geliştirildi.}
  \end{cvitems}
}
%---------------------------------------------------------
\cventry{Robotik Yazılım Mühendisliği Stajı}
{Triodor AR-GE Yazılım ve Bilişim}
{İstanbul, Türkİye}
{Temmuz 2015 - Ağustos 2015}{
  \begin{cvitems}
    \item {Ambar ve Besleme Mühendislik departmanında Juno Next robotunu dizayn eden takımda çalışıldı.}
    \item {V-REP simülasyon ortamında Juno Next prototipinin simülasyon modeli geliştirildi.}
    \item {Robot İşletim Sistemi (ROS) kullanılarak robotun kontrol sistemi yazıldı.}
    \item {Juno Next robotunun fonksiyonelitesini ve ROS sistemini test eden simülasyon senaryoları geliştirildi.}
  \end{cvitems}
}
%---------------------------------------------------------
\end{cventries}
%-------------------------------------------------------------------------------
%	SECTION TITLE
%-------------------------------------------------------------------------------
\cvsection{Yetenekler}
%-------------------------------------------------------------------------------
%	CONTENT
%-------------------------------------------------------------------------------
\begin{cvskills}
%---------------------------------------------------------
  \cvskill
    {Programlama} % Category
    {Python, C++ ,Scala, JavaScript, Rust, Matlab ve Java.} % Skills  

%---------------------------------------------------------
  \cvskill
    {Makine Öğrenmesi} % Category
  { Tensorflow ve Keras.} % Skills

%---------------------------------------------------------
  \cvskill
    {DevOps} % Category
    {AWS, Docker Jenkins ve CircleCI.} % Skills

%---------------------------------------------------------
  \cvskill
    {Back-end} % Category
    {Django ve REST API.} % Skills

%---------------------------------------------------------
  \cvskill
    {Diğer} % Category
    {Agile yazılım geliştirme, problem çözümü, iletişim, takım çalışması ve Latex.} % Skills


%---------------------------------------------------------
\end{cvskills}

\end{document}



% \cvsection{Summary}

%-------------------------------------------------------------------------------
%	CONTENT
%-------------------------------------------------------------------------------
\begin{cvparagraph}

Currently working as a Senior Software Engineer at \coloredHref{https://www.commencis.com}{Commencis} focusing on mobile application development. Graduated from Politecnico di Milano with Master of Science in Computer Science and Engineering. Interested in cross platform development, declarative programming frameworks, and scalable and modularized system design. Curious about web3 and blockchain technologies.
\end{cvparagraph}

% Old one
% Currently working as a Software Engineer at Commencis. 
% Graduated from Politecnico di Milano with Master of Science in Computer Science and Engineering. Interested % in scalable system design, functional programming and cloud computing. Enjoys working on a more optimized % % problem-solving methods for challenging tasks and learning new technologies and tools if the need arises. 
%
% \cvsection{Education}
%-------------------------------------------------------------------------------
%	CONTENT
%-------------------------------------------------------------------------------
\begin{cventries}
%---------------------------------------------------------
  \cventry
    {Master of Science in Computer Science and Engineering} % Degree
    {Politecnico di Milano} % Institution
    {Milano, Italy} % Location
    {Sep. 2017 - Jul. 2019} % Date(s)
    {
      \begin{cvitems} % Description(s) bullet points
        \item{\textbf{Thesis}: \coloredHref{http://hdl.handle.net/10589/149395}{Adversarially Learned Anomaly Detection Using Generative Adversarial Networks},\quad  Advisor: \coloredHref{https://boracchi.faculty.polimi.it/}{Giacomo Boracchi}}
        \item {\textbf{Courses Taken}: Deep Learning, Data and Text Mining, Soft Computing, Natural Language Processing, Recommender Systems, Computer Vision, Parallel Programming, Computer Security, Computing Infrastuctures, Model Identification and Data Analysis}
      \end{cvitems}
    }

%---------------------------------------------------------
%---------------------------------------------------------
\cventry
{Bachelor of Science in Computer Engineering } % Degree
{Bogazici University} % Institution
{Istanbul, Turkey} % Location
{Sep. 2012, Jun. 2017} % Date(s)
{
  \begin{cvitems} % Description(s) bullet points
    \item{\textbf{Graduation Project} : Image Parody Detection}
    \item {\textbf{Courses taken}:	Introduction to Robotics,Principles of Artificial Intelligence,
    Concurrency and Distributed Programming, Systems Programming, Machine Learning, Bayesian Statistics \& Machine Learning}
  \end{cvitems}
}

%---------------------------------------------------------

\end{cventries}

% \input{resume/experience.tex}
% \cvsection{Skills}


%-------------------------------------------------------------------------------
%	CONTENT
%-------------------------------------------------------------------------------
\begin{cvskills}

%---------------------------------------------------------
  \cvskill
    {Programming} % Category
    {\textbf{Language}: Proficient in Kotlin, Java and Python ; experience in Scala, Go, JavaScript, Rust \linebreak \textbf{Other}: Design Patterns, MVP \& MVVM Architectures, Reactive Programming} % Skills  

%---------------------------------------------------------

%---------------------------------------------------------
  \cvskill
    {Machine Learning} % Category
  { Tensorflow, Keras} % Skills

%---------------------------------------------------------
  \cvskill
    {DevOps} % Category
    {Experience in AWS, Docker, Jenkins \& CircleCI} % Skills

% %---------------------------------------------------------
%   \cvskill
%     {Back-end} % Category
%     {Experience in Django \& REST API} % Skills

%---------------------------------------------------------
  \cvskill
    {Other} % Category
    { Agile Software Developement, Jira, Git, Gerrit, Problem Solving, Communication, Teamwork, Latex} % Skills


%---------------------------------------------------------
\end{cvskills}
% \input{resume/honors.tex}
% \input{resume/presentation.tex}
% \input{resume/writing.tex}
% \input{resume/committees.tex}
%\input{resume/extracurricular.tex}


%-------------------------------------------------------------------------------